\documentclass[11pt]{article}
 \renewcommand{\familydefault}{\sfdefault}

\usepackage{listings}
\usepackage{xcolor}
\lstset{
    language=SQL, % Set the language to SQL
    basicstyle=\ttfamily\color{black}, % Regular font style
    commentstyle=\color{green!60!black}, % Set the color for comments
    keywordstyle=\bfseries\color{blue}, % Bold and color for keywords
    stringstyle=\color{orange}, % Set the color for strings
}

\usepackage{listings}
\usepackage{tabularx}
\usepackage{changepage}
\usepackage{amsfonts}
\usepackage{float}
\usepackage{amsmath}
\usepackage{graphicx}
\usepackage{hyperref}
\usepackage{caption}
\usepackage{fancyhdr}
\usepackage{geometry}
	\geometry{height=24 cm}
	\geometry{left=2.5 cm}
	\geometry{right=2.5 cm}
	\geometry{top=2 cm}
	\geometry{headheight=1 cm}

\setcounter{secnumdepth}{2}
\linespread{1}
\renewcommand{\labelitemi}{-}
\pagestyle{empty}


\begin{document}

\begin{center}
	\huge{\textbf{ECloth Mountain}}
\end{center}

\section{Abstract}
"ECloth Mountain" è un'azienda speccializzata nella vendita di abbigliamento 
tecnico da montagna, ha appena aperto il proprio sito di e-commerce e ha
commissionato il proprio database. \\
Dal sito di "ECloth Mountain" è possibile registrarsi e creare un account, per
salvare i propri dati e per poter effettuare acquisti. Inoltre è possibile
creare varie liste di prodotti, per poterli salvare e visualizzare in un secondo
momento. \\
L'utente per registrarsi deve inserire i propri dati personali, mentre
l'indirizzo e il metodo di pagamento non sono obbligatori. L'indirizzo e il
metodo di pagamento sono necessari al momento dell'acquisto, ma l'utente può 
decidere di non salvarli. L'utente attraverso il proprio account può rivedere 
gli acquisti effettuati e le liste di prodotti salvate.

\section{Analisi dei requisiti}

\subsection{Descrizione testuale}

Nella base di dati sono presenti i dati degli \textbf{utenti}, che si registrano
sul sito per effettuare gli acquisti. Di ogni utente sono noti: 

\begin{itemize}
	\item nome
	\item cognome
	\item email
	\item password
\end{itemize}

Ogni utente può memorizzare i dati di una \textbf{carta di credito}. Di ogni
carta di credito sono noti:

\begin{itemize}
	\item numero della carta
	\item intestatario della carta
	\item data di scadenza
	\item codice di sicurezza (CVV)
\end{itemize}

Ogni utente può memorizzare i dati di vari indirizzi di spedizione. Di ogni
\textbf{indirizzo} sono noti:

\begin{itemize}
	\item via
	\item numero civico
	\item città
	\item CAP
	\item stato
\end{itemize}

Ogni utente può avere diversi carrelli. Di ogni \textbf{carrello} sono noti:

\begin{itemize}
	\item nome del carrello
	\item costo totale
	\item data di creazione
	\item data di ultima modifica
	\item prodotti contenuti
\end{itemize}

Ogni utente può effettuare un \textbf{ordine}, effettuando il \textit{checkout}
del carrello desiderato. Di ogni ordine sono noti:

\begin{itemize}
	\item numero dell'ordine
	\item costo totale
	\item data di creazione
	\item data di ultima modifica
	\item prodotti contenuti
	\item provider del pagamento
	\item status dell'ordine
	\item indirizzo di spedizione
\end{itemize}

I pagamenti sono gestiti mediante un provider esterno, che si occupa di
effettuare il pagamento e di notificare il sito di "ECloth Mountain" del
risultato dell'operazione. \\
Gli utenti possono acquistare dei prodotti, presenti sul sito. Di ogni
\textbf{prodotto} sono noti:

\begin{itemize}
	\item nome
	\item descrizione
	\item prezzo
	\item quantità
	\item tag
\end{itemize}

Non è possibile acquistare un prodotto se la quantità disponibile è minore di
quella richiesta. 

\begin{table}[H]
	\centering
	\begin{tabularx}{\textwidth}{|l|X|l|}
		\hline
		\textbf{Termine} & \textbf{Descrizione} & \textbf{Collegamenti} \\
		\hline
		Utente & Persona che si registra sul sito & Carta di credito,
		Lista, Indirizzo \\
		\hline
		Carta di credito & Carta di credito di un utente & Utente \\
		\hline
		Indirizzo & Indirizzo di spedizione di un utente & Utente,
		Ordine \\
		\hline
		Lista & Lista di prodotti, con relativo costo totale & Utente,
		Prodotto \\
		\hline
		Carrello & Lista con un nome & Entità figlia di Lista \\
		\hline
		Ordine & Lista con un identificativo e i dettagli di pagamento
		& Indirizzo, Entità figlia di Lista \\
		\hline
		Prodotto & Prodotto in vendita sul sito & Lista, Categoria \\
		\hline
		Categoria & Categoria di prodotti & Prodotto \\
		\hline
	\end{tabularx}
	\caption{Glossario dei termini}
\end{table}

\begin{table}[H]
	\centering
	\begin{tabular}{|l|l|l|}
		\hline
		\textbf{Operazione} & \textbf{Tipo} & \textbf{Frequenza} \\
		\hline
		Inserimento di un nuovo Utente & U & 1 volta al giorno \\
		\hline
		Inserimento di una Carta di Credito & U & 5 volte al giorno \\
		\hline
		Inserimento di un Indirizzo & U & 20 volte al giorno \\
		\hline
		Inserimento di un nuovo Carrello & U & 20 volte al giorno \\
		\hline
		Cancellazione di un Carrello & U & 10 volte al giorno \\
		\hline
		Visione di un Carrello & U & 100 volte al giorno \\
		\hline
		Aggiungi un Prodotto al Carrello & U & 100 volte al giorno \\
		\hline
		Rimuovi un Prodotto dal Carrello & U & 90 volte al giorno \\
		\hline
		Inserimento di un nuovo Ordine & U & 10 volte al giorno \\
		\hline
		Cancellazione di un Ordine & U & 1 volte al giorno \\
		\hline
		Visione di un Ordine & U & 10 volte al giorno \\
		\hline
		Inserimento di un nuovo Prodotto & B & 10 volte alla settimana \\
		\hline
		Calcolare il guadagno mensile & B & 1 volta al mese \\
		\hline
		Calcolare il guadagno quotidiano & B & 1 volta al giorno \\
		\hline
	\end{tabular}
	\caption{Operazioni}
\end{table}

\section{Progettazione concettuale}

\renewcommand{\labelitemi}{\textbullet}
\subsection{Lista delle entità}
\begin{itemize}
	\item Utente
		\begin{itemize}
			\item \underline{email} VARCHAR(255) PK NOT NULL
			\item password VARCHAR(255) NOT NULL
			\item nome VARCHAR(255) NOT NULL
			\item cognome VARCHAR(255) NOT NULL
		\end{itemize}

	\item Carta di credito
		\begin{itemize}
			\item \underline{numero} VARCHAR(16) PK NOT NULL
			\item intestatario VARCHAR(255) NOT NULL
			\item scadenza DATE NOT NULL
			\item cvv INT NOT NULL
		\end{itemize}

	\item Indirizzo
		\begin{itemize}
			\item \underline{via} VARCHAR(255) PK NOT NULL
			\item \underline{numero} PK INT NOT NULL
			\item \underline{cap} VARCHAR(5) PK NOT NULL
			\item citta VARCHAR(255) NOT NULL
			\item stato VARCHAR(255) NOT NULL
		\end{itemize}

	\item Lista
		\begin{itemize}
			\item \underline{id utente} INT PK NOT NULL
			\item \underline{data creazione} DATE PK NOT NULL
			\item data modifica DATE NOT NULL
			\item costo totale FLOAT NOT NULL
		\end{itemize}

		\begin{adjustwidth}{2em}{0pt}

			L'entità Lista si specializza in due sottocategorie con una 
			generalizzazione totale:

			\renewcommand{\labelitemii}{\textbullet}
			\renewcommand{\labelitemiii}{-}
			\renewcommand{\labelitemiv}{$\square$}


			\begin{itemize}
				\item Carrello
					\begin{itemize}
						\item nome VARCHAR(255) NOT NULL
					\end{itemize}

				\item Ordine
					\begin{itemize}
						\item numero INT NOT NULL

						\item dettagli pagamento: attributo composto:
							\begin{itemize}
								\item provider VARCHAR(255) NOT NULL
								\item status VARCHAR(255) NOT NULL
							\end{itemize}

					\end{itemize}
			\end{itemize}

		\end{adjustwidth}

	\item Prodotto
		\begin{itemize}
			\item \underline{nome} VARCHAR(255) PK NOT NULL
			\item \underline{tag} VARCHAR(255) PK NOT NULL
			\item descrizione VARCHAR(255)
			\item prezzo FLOAT NOT NULL
			\item quantità INT NOT NULL
		\end{itemize}

	\item Categoria
		\begin{itemize}
			\item \underline{nome} VARCHAR(255) PK NOT NULL
			\item descrizione VARCHAR(255)
		\end{itemize}
\end{itemize}

\begin{table}[H]
	\centering
	\begin{tabularx}{\textwidth}{|l|X|X|l|}
		\hline 
		\textbf{Relazione} & \textbf{Entità coinvolte} & \textbf{Descrizione} &
		\textbf{Attributi} \\
		\hline
		Metodo pagamento & Carta di credito (1, 1), Utente (0, 1) & Un utente
		può avere al più una carta di credito, una carta di credito si riferisce
		ad un solo cliente & Nessuno \\
		\hline
		Indirizzo spedizione & Indirizzo (0, N), Utente (0, N) & Un utente può
		avere più indirizzi di spedizione, o non averne uno affatto. Un
		indirizzo può essere associato a più clienti e può non avere nessun
		cliente & Nessuno \\
		\hline
		& Lista (1, 1), Utente (0, N) & Un utente può avere più liste, o non
		averne affatto. Una lista si riferisce ad un utente & Nessuno \\
		\hline
		Contenuto & Lista (0, N), Prodotto (0, N) & Una lista può contenere più
		prodotti, un prodotto può essere contenuto in più liste & quantità INT
		$>$ 0 \\
		\hline
		Spedizione & Ordine (1, 1), Indirizzo (0, N) & Un ordine può essere
		spedito ad un solo indirizzo. Un indirizzo può essere associato a più
		ordini & Nessuno \\
		\hline
		Tag & Prodotto (1, 1), Categoria (0, N) & Un prodotto appartiene ad una
		categoria, una categoria può avere più prodotti & Nessuno \\
		\hline
	\end{tabularx}
	\caption{Tabella delle relazioni}
\end{table}

\subsection{Schema Concettuale}

\subsubsection{Vincoli non rappresentabili tramite schema E-R:}
\begin{itemize}
	\item Un indirizzo è associato ad un utente o ad un ordine (o non
		esclusivo).

\end{itemize}

\subsubsection{Vincoli di derivazione:}
\begin{itemize}
	\item L'attributo costo di una lista è dato dalla somma dei prezzi dei
		prodotti contenuti nella lista moltiplicati per la quantità di ciascuno
		di essi.
\end{itemize}

\begin{figure}[H]
	\centering
	\includegraphics[width=\textwidth]{schema_concettuale}
	\caption{Schema concettuale}
\end{figure}

\section{Progettazione logica}

\subsection{Ristrutturazione}

\subsubsection{Analisi delle ridondanze}

L'attributo "Costo" dell'entità Lista è ridondante, in quanto può essere
calcolato come somma dei prezzi dei prodotti contenuti nella lista moltiplicati
per la quantità di ciascuno di essi. L'attributo "Costo" viene utilizzato nelle
operazioni:
\begin{enumerate}
	\item Aggiungi Prodotto al Carrello
	\item Rimuovi Prodotto dal Carrello
	\item Visualizza Carrello
	\item Visualizza Ordine
\end{enumerate}

Poichè le operazioni 1 e 2 modificano l'attributo "Costo" dell'entità Lista e
entrambe queste due operazioni sono molto più frequenti dell'operazione 3, si
toglie l'attributo "Costo" dall'entità Lista e si calcola il costo della lista
ogni volta che viene richiesta l'operazione 3. Si potrebbe pensare di aggiungere
l'attributo "Costo" all'entità Ordine, ma l'operazione 4 è poco frequente.

\begin{table}[H]
	\centering
	\begin{tabularx}{\textwidth}{|l|X|}
		\hline
		\textbf{Generalizzazione} & \textbf{Risoluzione} \\
		\hline
		Lista $\leftarrow$ Carrello, Ordine & L'entità Carrello è accorpata in
		Lista, con l'aggiunta di un attributo "nome VARCHAR(255) not null".
		L'entità Ordine è rappresentata mediante una relazione tra Lista (0,1) e
		Indirizzo (0, N), dove gli attributi di Ordine diventano attributi della
		relazione. \\
		\hline
	\end{tabularx}
	\caption{Eliminazione delle generalizzazioni}
\end{table}

\subsubsection{Scelta degli identificatori primari}
È stato introdotto l'attributo "ID" per le entità Prodotto, Indirizzo e Lista, 
dunque è stato rimosso l'attributo "Numero" dall'entità Ordine.


\begin{figure}[H]
	\centering
	\includegraphics[width=\textwidth]{schema_logico}
	\caption{Schema E-R ristrutturato}
\end{figure}

\subsubsection{Creazione delle tabelle}
$A \rightarrow B$ indica che B è chiave esterna di A.

\paragraph{Carta di credito}(\underline{numero}, intestatario, scadenza,
CVV, utente $\rightarrow$ Utente.email)

\paragraph{Utente}(\underline{email}, password, nome, cognome)

\paragraph{Indirizzo spedizione}(\underline{utente} $\rightarrow$ Utente.email,
\underline{indirizzo} $\rightarrow$ Indirizzo.ID)

\paragraph{Indirizzo}(\underline{ID}, via, numero civico, CAP, città, stato)

\paragraph{Ordine}(\underline{lista} $\rightarrow$ Lista.ID, 
indirizzo $\rightarrow$ Indirizzo.ID, status, provider)

\paragraph{Lista}(\underline{ID}, nome, utente $\rightarrow$ Utente.email, data
creazione, data modifica)

\paragraph{Contenuto}(\underline{lista} $\rightarrow$ Lista.ID,
\underline{prodotto} $\rightarrow$ Prodotto.ID, quantità)

\paragraph{Prodotto}(\underline{ID}, nome, descrizione, prezzo, quantità, tag
$\rightarrow$ Categoria.nome)

\paragraph{Categoria}(\underline{nome}, descrizione)


\section{Here I write the queries}

\begin{lstlisting}[language=SQL]
SELECT lista.utente, sum(contenuto.quantita)
FROM lista, contenuto
WHERE lista.id = contenuto.lista 
AND lista.data_creazione > '2023-01-01'
GROUP BY (lista.utente);
\end{lstlisting}





\begin{lstlisting}[language=SQL]
SELECT utente.email, count(utente.email)
FROM utente, lista, ordine
WHERE utente.email = lista.utente
AND lista.id = ordine.lista
GROUP BY (utente.email)
\end{lstlisting}




\begin{lstlisting}[language=SQL]
SELECT email, count(email) 
FROM ((
	SELECT utente.email, lista.id from utente, lista
	WHERE utente.email = lista.utente) as tab
	LEFT OUTER JOIN ordine
	ON tab.id = ordine.lista)
WHERE lista IS NULL
GROUP BY email
\end{lstlisting}




\begin{lstlisting}[language=SQL]
SELECT ordine.lista, 
	sum(prodotto.prezzo * contenuto.quantita) as totale
FROM ordine, prodotto, contenuto
WHERE ordine.lista = contenuto.lista
AND prodotto.id = contenuto.prodotto
GROUP BY ordine.lista
\end{lstlisting}




\begin{lstlisting}[language=SQL]
SELECT lista.nome, totale 
FROM (
	SELECT tab.id, sum(prezzo * quantita) as totale
	FROM (
		SELECT lista.id as id, prodotto.prezzo, 
			contenuto.quantita
		FROM lista, prodotto, contenuto
		WHERE lista.id = contenuto.lista
		AND prodotto.id = contenuto.prodotto
	) as tab 
	LEFT JOIN ordine
	ON tab.id = lista
	WHERE indirizzo is NULL
	GROUP BY tab.id) as tab2, lista
WHERE tab2.id = lista.id;
\end{lstlisting}




\begin{lstlisting}[language=SQL]
SELECT DATE(data_creazione) AS day, sum(totale)
FROM lista, (
	SELECT ordine.lista, 
		sum(prodotto.prezzo * contenuto.quantita) AS totale
	FROM ordine, prodotto, contenuto
	WHERE ordine.lista = contenuto.lista
	AND prodotto.id = contenuto.prodotto
	GROUP BY ordine.lista) as costo_ordini
WHERE lista.id = lista
GROUP BY day;
\end{lstlisting}





\begin{lstlisting}[language=SQL]
SELECT EXTRACT(YEAR from data_creazione) AS year, 
	EXTRACT(MONTH from data_creazione) AS month, sum(totale)
FROM lista, (
	SELECT ordine.lista, 
		sum(prodotto.prezzo * contenuto.quantita) AS totale
	FROM ordine, prodotto, contenuto
	WHERE ordine.lista = contenuto.lista
	AND prodotto.id = contenuto.prodotto
	GROUP BY ordine.lista) AS costo_ordini
WHERE lista.id = lista
GROUP BY year, month
ORDER BY year, month;
\end{lstlisting}





\begin{lstlisting}[language=SQL]
SELECT p.tag, sum(p.prezzo * c.quantita) as totale
FROM prodotto as p, contenuto as c, ordine as o
WHERE p.id = c.prodotto
AND o.lista = c.lista
GROUP BY p.tag
HAVING sum(p.prezzo * c.quantita) > 100
ORDER BY totale DESC;
\end{lstlisting}

\section{Esecuzione queries in codice in cpp}
\subsection{Strutturazione}

Per l'esecuzione delle queries precedenti, si è scelto di scrivere un singolo file cpp contenente il codice in grado di svolgere le seguenti funzionalità:   
\begin{itemize}
	\item connessione al database.
	\item esecuzione di ogni interrogazione.
	\item stampa dei risultati per una generica interrogazione.
\end{itemize}

Ogni interrogazione è stata messa all'interno di una procedura avente la firma query(PGconn*); all'interno di essa si trova il codice necessario per poterla eseguire la query e stamparne i risultati.

\subsection{Esecuzione}

Per poter eseguire una delle interrogazioni disponibili, all'interno del main viene creata una mappa di puntatori a funzione aventi la firma query(PGconn*); in particolare la mappa è del tipo: std::map<unsigned int, void (*)(PGconn*)>.

Viene poi creata una mappa del tipo std::map<unsigned int, std::string> contenente le singole descrizioni del risultato di ogni query.

A questo punto viene instaurata la connessione al database mediante la funzione checkConnection(PGconn*); nel caso in cui la connessione sia andata a buon fine, si entra in un ciclo do while che esegue le seguenti istruzioni:

\begin{enumerate}
	\item stampa della mappa contenente le descrizioni delle query.
	\item viene chiesto all'utente di immettere un numero di quelli specificati dalle descrizioni precedentemente stampate.
	\item tramite il numero ottenuto, viene eseguita la query associata ad esso mediante la mappa di puntatori a funzine
\end{enumerate}

Il codice contenuto nel main è il seguente.

\begin{lstlisting}
int main() {
	std::map<unsigned int, std::string> messaggi;
	// inizializzazione messaggi
	std::map<unsigned int, void (*)(PGconn*)> queries;
	// inizializzazione queries 
	
	
	// SETUP CONNESSIONE al DB
	PGconn* conn;
	char conninfo [250];
	sprintf(conninfo, 
	"user=%s password=%s dbname=%s hostaddr=%s port=%d" ,
	PG_USER , PG_PASS , PG_DB , PG_HOST , PG_PORT ) ;
	
	// APERTURA DELLA CONNESSIONE
	conn = PQconnectdb(conninfo);

	// CONTROLLO DELLA CONNESSIONE
	checkConnection(conn);


	unsigned int operazione = 0;
	do {
		print_messages(messaggi);
		std::cin >> operazione;
		if (operazione != 0) {
			queries[operazione](conn);
		}
	} while(operazione != 0);
} 
\end{lstlisting}

\end{document}














